 \chapter{Estado del Arte}
\label{chap:estado-arte}

\lettrine{E}{l} registro de la información comercial no es algo nuevo y la forma de tratar esta información ha ido evolucionándo. Y como no podía ser de otra forma la popularización de los sistemas de información supuso un salto cuantitativo a la hora de gestionar todos esos datos, desde las incipientes automatizaciones a través de las bases de datos de los 60 hasta los productos \acrshort{crm} y \acrshort{erp} comercializados hoy en día.

Existen multitud de \acrshort{crm} y \acrshort{erp} en el mercado. Nos hemos centrado en cuatro de ellas que por sus características consideramos las más relevantes, bien por su posición en el mercado o por ser una alternativa de \gls{opensource} con fuerte presencia en el sector.



\section{SalesForce CRM}
\label{sec:estado-arte-salesforce}

Salesforce CRM~\cite{SalesforceCRM} es un \acrfull{saas} por lo que, al residir en la nube permite el acceso la información en cualquier momento y desde cualquier dispositivo. Se trata de un \gls{swpropietario} cuyo modelo de negocio se basa en suscripciones de pago por uso.

El proyecto de Salesforze arrancó en 1999 y actualmente tiene su sede central en San Francisco, California (EE.UU). Fueron los primeros en lanzar un \acrshort{crm} basado en la nube y a día de hoy más de 150.000 empresas de todos los ámbitos y tamaños utilizan este producto, siendo actualmente la empresa que posee la mayor cuota de mercado de los \acrshort{crm}. \newline

Entre los SaaS ofrecidos por el sistema Salesforce se encuentra un \acrshort{crm}, una plataforma de márketing digital, una plataforma de servicio orientada al equipo de servicio que se complementa con redes sociales, una plataforma de comunicación para empleados, socios y clientes, una plataforma de análisis y visualización de dtos y una plataforma para desarrollar aplicaciones personalizadas que se ejecutan en la plataforma Salesforce.

Entre las ventajas que ofrece se encuentran las siguientes:

\begin{itemize}
\item Facilidad de uso.
\item Consta de una amplia gama de funcionalidades.
\item Es muy personalizable.
\item Es altamente escalable.
\item Basado en tecnologías cloud.
\item Su implementación es rápida.
\end{itemize}

Entre las principales desventajas de esta solución destacan:

\begin{itemize}
\item Precio. El paquete completo de Salesforce CRM no está al alcance de todos y muchas pequeñas empresas no pueden costearlo.
\item Su exceso de funcionalidad puede suponer un inconveniente para empresas pequeñas sin equipos de ventas o marketing por ejemplo, que no podrán sacar todo el provecho a la herramienta.
\item Presenta un elevado número de actualizaciones que impactan en la interfaz, lo que en ocasiones confunde al usuario.
\end{itemize}


\section{SAP S/4HANA}
\label{sec:estado-arte-sap}

SAP~\citep{SAP} es una empresa fundada en 1972 bajo el nombre de \textit{Systemanalyse Programmentwicklung} (Desarrollo de programas de sistemas de análisis) que actualmente tiene su sede en Walldorf, Baden-Württemberg (Alemania). Con la presentación de su software original SAP R/2 y SAP R/3, SAP estableció el estándar global para el \acrlong{erp}.

En 2015 se produjo el lanzamiento de SAP S/4HANA Cloud, un software modular y completo potenciado por la inteligencia artificial y el \textit{machine learning} o aprendizaje automático. Se trata de un \gls{swpropietario} que ha sido diseñado como una solución \acrshort{erp} escalable a pequeñas, medianas y grandes empresas en función de sus objetivos y necesidades, siendo válido tanto para entornos orientados en la nube como on-premise. Esta solución es la que lidera actualmente el mercado en europa occidental. \newline

SAP S/4HANA abarca las siguientes áreas de negocio: finanzas, gestión de activos, abastecimiento y adquisiciones, gestión de la cadena de suministros, I+D e ingeniería 


Entre las ventajas de este \acrshort{erp} se encuentran las siguientes:
\begin{itemize}
\item Es personalizable y funcional.
\item Es portable.
\item Es compatible con otras aplicaciones gracias a su programación basada en módulos funcionales.
\item Consta de una capacidad estadística bien integrada.
\end{itemize}


En cuanto a los inconvenientes de este software destacan lo siguiente:

\begin{itemize}
\item Una elevada curva de aprendizaje.
\item La implementación por módulos resulta lenta ya que suele llevarse a cabo de forma secuencial.
\item Tiene un elevado coste debido a que requiere de equipos potentes para su ejecución.
\item Tiene un elevado número de actualizaciones que suelen acarrear costes (en seguridad, almacenamiento, \dots).
\item Se requiere un gasto adiciona de personal específico para su mantenimiento.
\end{itemize}


\section{ODOO}
\label{sec:estado-arte-odoo}

Odoo~\cite{Odoo} es una empresa nacida en 2005 a partir del proyecto de de software de gestión empresarial en \gls{opensource} TinyERP cuya sede central se encuentra en Ramillies, Bélgica.

Su software Odoo es un \acrshort{erp} integrado que cuenta con una versión \textit{comunitaria} de código abierto bajo licencia LGPLv3 y una versión empresarial bajo licencia comercial que complementa la edición comunitaria y da acceso a todas las funcionalidades del producto. Su política de precios se basa en el número de usuarios y número de módulos instalados.

Se trata de un software empresarial formado por un conjunto perfectamemte integrado de aplicaciones que incluye módulos para gestión de proyectos, de almacenes, ventas, comercio electrónico, facturación y contabilidad entre otros. Está disponible tanto para entornos orientados en la nube como on-premise.

Entre las ventajas de esta solución empresarial destacan las siguientes:

\begin{itemize}
\item Sus multiples aplicaciones permiten cubrir todas las áreas de negocio de una empresa.
\item Su modularidad permite adaptar sus productos a las necesidades particulares de cada empresa
\item Su política de precios basada en el número de usuarios y módulos instalados permite \textit{pagar por lo que realmente se usa}, por lo que puede ser una gran alternativa para pequeñas empresas que no necesitan gran cantidad de aplicaciones para gestionar su negocio.
\item Cuenta con una gran comunidad de usuarios que facilita mucho la resolución de dudas. 
\item Su filosofía de \gls{opensource} lo convierte en una alternativa altamente segura, ya que es más fácil detectar vulnerabilidades en el código y corregirlas.
\end{itemize}

Algunos de los inconvenientes que presenta este producto son:

\begin{itemize}
\item Un precio mucho mayor del esperado si se requieren muchos extras o aplicaciones.
\item No existe la funcionalidad de compatibilidad de versiones, por lo que una nueva versión del producto requiere una migración de datos.
\end{itemize}



\section{ERPNext}
\label{sec:estado-arte-erpNext}
ERPNext~\cite{ERPNext} es un software \acrshort{erp} desarrollado por Frappe Technologies Pvt. Ltd, con sede en Maharashtra, India. Es un software \gls{opensource} desarrollado bajo licencia LGPLv3.

Se trata de un \acrshort{saas} configurable que consta de diferentes módulos como pueden ser \acrshort{crm}, gestión de inventario, de suscripciones, de proyectos, ventas, facturación y atención al cliente entre otros. Consta de tres planes de suscripción mensual basados principalmente en el número de usuarios y requisitos de almacenamiento necesarios.

Entre sus ventajas destacan:
\begin{itemize}
\item No existen costes de licencia.
\item La descarga es gratuita, sólo se paga por el soporte de software.
\item Es flexible y permite adaptarse a cualquier tipo de empresa, pudiendo utilizar sólo aquellos módulos que se adaptan a las necesidades de la empresa.
\item Permite la migración de datos desde el sistema local.
\end{itemize}

Entre los inconvenientes que presenta se encuentran:
\begin{itemize}
\item No es adecuado para grandes empresas.
\item Problemas de actualización con versiones personalizadas.
\item Gestión de permisos compleja.
\end{itemize}


\section{Conclusión}
\label{sec:estado-arte-conclusion}
Como hemos visto las herramientas \acrshort{crm}/\acrshort{erp} que dominan el mercado son de \gls{swpropietario} que ofrecen soluciones muy potentes que pueden ser muy ventajosas para grandes empresas, pero no así para empresas más pequeñas, las cuales no siempre se pueden permitir el coste que este tipo de software conlleva o, en caso de poder, acaban infrautilizando el software por el que están pagando, ya que las funcionalidades mínimas contratadas suelen exceder las necesidades reales de la empresa. Para estos existen herramientas de \gls{opensource} con gran presencia en el mercado en el mercado que pueden ser una buena alternativa a las soluciones de software privativo, aunque tienen el inconveniente de que suelen presentar problemas de compatibilidad de versiones.
