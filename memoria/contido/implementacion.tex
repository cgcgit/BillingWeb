\chapter{Implementación}
\label{chap:implementacion}


\lettrine{E}{n} este capítulo se detallan los detalles técnicos necesarios para llevar a cabo la implementación de la aplicación.


\section{Hardware Utilizado}
\label{sec:hw}
El proyecto fue desarrollado sobre un equipo con las siguientes características:
\begin{itemize}
\item Procesador: Intel® Core™ i7-1195G7, 2.90GHz × 8
\item Memoria: 16,0 GiB
\item Disco Duro: 1,0 TB
\end{itemize}
	

\section{Software Utilizado}
\label{sec:sw}	

 UBUNTU 22.04.1 LTS y 22.04.1 LTS\newline
 Sistema operativo sobre el que se desarrolló la aplicación.
\newline
 
 WILDFLY 20.0.1\newline
 Servidor de aplicaciones \acrshort{jee} de \gls{opensource}, utilizado para la implantación de la aplicación web.
 \newline
 
 MOZILLA FIREFOX 104.0\newline
 Navegador web usado para la ejecución y pruebas de la aplicación web.
\newline
 
 ECLIPSE 4.16 2020-06\newline
 \acrfull{ide} \gls{opensource} utilizado para la generación del código de la aplicación .
\newline
 
 JAKARTA EE 8.0 (y JAVA SE 1.8.0)\newline
 Lenguaje de programación utilizado para desarrollar la aplicación.
\newline
 
 MAVEN 3.6.3\newline
 Herramienta de software de \gls{opensource} para la gestión y construcción de proyectos Java.
 \newline

 
 POSTGRESQL 14.5\newline
 \acrshort{sgbd} de \gls{opensource} utilizado como soporte de almacenamiento.
\newline
 
 DBeaver 22.1.4\newline
 Herramienta gráfica \gls{opensource} de diseño y gestión de bases de datos, utilizada para gestionar la base de datos creada para la aplicación.
 \newline

 SCHEMASPY 5.0.0\newline
 Herramienta basada en Java que genera representaciones visuales de un esquema de base de datos en un formato visible por el navegador.
 \newline
 
 UMLET 15.0\newline
 Herramienta \gls{opensource} para la modelización de diagramas \acrshort{uml}.
\newline

 
 UMBRELLO 2.35.0\newline
 Programa para desarrllollar diagramas en \acrfull{uml} basado en tecnología KDE.
 \newline
 
 \LaTeX{}\newline
 Sistema de composición de textos utilizado para generar el presente documento.
 \newline
 
 TEXMAKER 5.0.3\newline
 Editor \LaTeX{} para redactar el presente documento.
 \newline
 
 
 

	
