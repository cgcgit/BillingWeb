\chapter{Implementación}
\label{chap:implementacion}


\lettrine{E}{n} este capítulo se detallan los detalles técnicos necesarios para llevar a cabo la implementación de la aplicación.


\section{Hardware Utilizado}
\label{sec:hw}
El proyecto fue desarrollado sobre un equipo con las siguientes características:
\begin{itemize}
\item Procesador: Intel® Core™ i7-1195G7, 2.90GHz × 8
\item Memoria: 16,0 GiB
\item Disco Duro: 1,0 TB
\end{itemize}
	

\section{Software Utilizado}
\label{sec:sw}	

\begin{itemize}
 \item UBUNTU~\cite{Ubuntu} 22.04.1 LTS y 22.04.1 LTS\newline
 Sistema operativo sobre el que se desarrolló la aplicación.
 \item WILDFLY~ \cite{Wildfly} 20.0.1\newline
 Servidor de aplicaciones \acrshort{jee} de \gls{opensource}, utilizado para la implantación de la aplicación web.
 \item MOZILLA~\cite{Mozilla} FIREFOX 104.0\newline
 Navegador web usado para la ejecución y pruebas de la aplicación web.
  \item CHROMIUM~\cite{Chromium} 105.0.5195.52\newline
 Navegador web usado para la ejecución y pruebas de la aplicación web.
 \item ECLIPSE~\cite{EclipseIDE} 4.16 2020-06\newline
 \acrfull{ide} \gls{opensource} utilizado para la generación del código de la aplicación.
 \item JAKARTA EE 8.0~\cite{JakartaEE} (y OpenJDK~\cite{OpenJDK} 1.8.0)\newline
 Lenguaje de programación utilizado para desarrollar la aplicación.
 \item APACHE MAVEN~\cite{ApacheMaven} 3.6.3\newline
 Herramienta de software de \gls{opensource} para la gestión y construcción de proyectos Java.
 \item PRIMEFACES~\cite{Primefaces} 11.0.0\newline
 \emph{Framework} de \gls{opensource} para \acrshort{jsf}.
 \item OMNIFACES~\cite{Omnifaces} 3.13.2\newline
 Biblioteca de utilidades open source para \acrshort{jsf}. 
 \item JOOQ~\cite{JooQ} 3.14.16\newline
 Biblioteca de utilidades open source para \acrshort{jsf}.  
 \item LOG4J~\cite{Log4j} 2.15.0\newline
 Biblioteca \gls{opensource} perteneciente a los Java Logging Frameworks 
 \item POSTGRESQL~\citep{Postgresql} 14.5\newline
 \acrshort{sgbd} de \gls{opensource} utilizado como soporte de almacenamiento.
 \item DBeaver~\cite{DBeaver} 22.1.4\newline
 Herramienta gráfica \gls{opensource} de diseño y gestión de bases de datos, utilizada para gestionar la base de datos creada para la aplicación.
 \item SCHEMASPY~\cite{SchemaSpy} 5.0.0\newline
 Herramienta basada en Java que genera representaciones visuales de un esquema de base de datos en un formato visible por el navegador. 
 \item UMLET~\cite{Umlet} 15.0\newline
 Herramienta \gls{opensource} para la modelización de diagramas \acrshort{uml}.
 \item  \LaTeX{}~\cite{Latex}\newline
 Sistema de composición de textos utilizado para generar el presente documento. 
 \item TEXMAKER~\citep{TexMaker} 5.0.3\newline
 Editor \LaTeX{} para redactar el presente documento.
\end{itemize} 

	
