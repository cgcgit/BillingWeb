\chapter{Metodología}
\label{chap:metodologia}

\lettrine{E}{n} informática se define la metodología como \textquote{un proceso para producir software de forma organizada, empleando una colección de técnicas y convenciones de notación predefinidas.}

Existen distintas metodologías de desarrollo de software. Para desarrolla este \acrshort{tfg} nos hemos basado en una metodología clásica basada en el método incremental, ya que se partía de una base de conocimiento del proceso de negocio previa pero no así de la tecnología a usar.

Nuestro modelo incremental ha constado de tres fases, cada una de las cuales ha contado con sus correspondiente parte de análisis, diseño, codificación y pruebas y ha culminado con una entrega del producto completamente operacional atendiendo a los requisitos especificados para dicha fase.

\section{Descripción de las fases definidas}
\label{sec:metodologia-fases}

A continuación se muestra una breve descripción para cada una de las fases definidas con el método incremental usado. Los casos de uso definidos en cada una de las fases se detallan en el anexo \ref{chap:ref-tecnica}.

\subsection{Fase I - Fase inicial}
Finalización estimada de la fase: finales de mayo.

En esta fase se realiza un análisis global del producto a desarrollar, realizando una primera aproximación a los elementos que deberá contener.
Se realiza un análisis de los requisitos técnicos necesarios para llevar a cabo el desarrollo y se establece el entorno de desarrollo, realizando las instalaciones y configuraciones pertinentes. 

Atendiendo a la parte funcional del producto a desarrollar, esta fase se centra en crear la estructura del proyecto: el acceso a la aplicación y la parte correspondiente a los elementos de parametrización y de catálogo simple, los elementos con menor complejidad de la aplicación. Se realiza la parte de análisis, diseño, codificación y pruebas correspondientes a los casos de uso \textbf{CU-02 Inicio de sesión}, \textbf{CU-02 Fin de sesión}, \textbf{CU-19 Gestionar ficha personal}, \textbf{CU-29 Gestión de usuarios}, \textbf{CU-22 Acceder parametrización}, \textbf{CU-23 Acceder catálogo entidad simple} y \textbf{CU-25 Acceder a relación simple del catálogo}, (con sus correspondientes casos de uso anidados). 


\subsection{Fase II - Elementos históricos}
Finalización estimada de la fase: mediados de julio.

Esta fase se centra en el desarrollo de las entidades con histórico del catálogo. Se realiza la parte de análisis, diseño, codificación y pruebas correspondientes a los casos de uso \textbf{CU-24 Acceder al catálogo entidad con históricos} y \textbf{CU-26 Acceder a relación con histórico del catálogo} (con sus correspondientes casos de uso anidados).


\subsection{Fase III - Contratación}
Finalización estimada de la fase: finales de agosto.

Fase final del proyecto que se centra en el desarrollo de las instancias del catálogo y la vista jerárquica de la contratación. Se realiza la parte de análisis, diseño, codificación y pruebas correspondientes a los casos de uso \textbf{CU-26 Acceder a relación con histórico del catálogo} y \textbf{CU-28 Acceder jerarquía} (con sus correspondientes casos de uso anidados). 



