\chapter{Pruebas de software}
\label{chap:pruebas}

El enfoque que se ha seguido para la validación del software desarrollado es la utilización de los casos de uso del actor READ y los específicos de los actores WRITE y ADMIN para cada una de las entidades existentes en el sistema. A partir de los mismos se han probado las distintas acciones que se
pueden realizar en el sistema y su correcto funcionamiento.

Aunque muchas pruebas se han ido realizando de forma paralela a la
codificación, se distinguen dos etapas donde se han intensificado, coincidiendo una con la obtención de un primer prototipo, que constituye el grueso de la aplicación, y otra con el final de la codificación.

Para probar el funcionamiento global se han introducido un conjunto de datos con las casuísticas suficientes como para permitir verificar todas las funcionalidades y tener una visión global del comportamiento de la aplicación una vez puesta en producción.


