\chapter{Conclusiones}
\label{chap:conclusiones}

\lettrine{E}{n} este \acrshort{tfg} se ha desarrollado una herramienta para la gestión del catálogo de servicios y contratación de una cartera de clientes para una empresa proveedora de servicios, alcanzándose los principales objetivos que se pretendían al comienzo de su realización:
\begin{itemize}
\item Se ha implementado una aplicación con acceso vía web que abarca los principales componentes que conforman el sistema: catálogo de servicios, contrataciones y la parametrización necesaria para la configuración de estos elementos.

\item La herramienta permite una gestión de usuarios para implementar el control de acceso o modificación de la información.

\item Se definen distintos niveles de acceso a las funcionalidades de la aplicación en función de distintos perfiles de usuario, lo que evita accesos inadecuados a determinadas funcionalidades del sistema.

\item A la hora de manejar entidade con histórico se realiza una correcta gestión de las fechas de inicio y fin de los mismos de forma que queda garantizada la no existencia de \textit{huecos} entre registros históricos para una misma entidad o instancia (es decir, todos los registros comprendidos entre la fecha mínima de inicio y la máxima de fin son consecutivos).

\item Permite gestionar las diferentes relaciones de dependencia de entidades del catálogo de servicios con un sólo click.

\item Permite búsquedas sencillas para establecer las relaciones de dependencia de las distintas contrataciones a realizar.

\item Contiene una vista jerárquica de las contrataciones realizadas por un cliente, de forma que se puede ver la relación de todos los componentes contratados de un vistazo.

\end{itemize}


A esto hay que sumar el cumplimiento de los objetivos personales de ampliar mis conocimientos en distintas áreas técnicas como la de iniciar una aplicación web desde cero con \acrshort{jee} o gestionar una base de datos completa.



\section{Líneas futuras}
\label{sec:futuro}

La herramienta aquí presentada es un prototipo cuyo principal objetivo es la puesta en escena de las principales funcionalidades descritas para la herramienta de contratación. Con vistas a la difusión de una herramienta como la propuesta sería interesante mejorar, o desarrollar en algunos casos, los siguientes aspectos:
\begin{itemize}
\item Mejorar el entorno gráfico para que resulte más agradable y atractivo redimensionando componentes, usando funcionalidades de \textit{drag and drop} (arrastrar y soltar) para el establecimiento de relaciones o la contratación de las distintas entidades.
\item Relacionar las distintas vistas a través de enlaces, de forma que pulsando dicho enlace se pueda acceder a otras vistas relacionadas.
\item Usar componentes en el establecimiento de las relaciones del catálogo y en la contratación de productos, cuotas y servicios, para facilitar el prodedim
\item Mejorar la experiencia de los usuarios añadiendo opciones de personalización de la interfaz.
\item Incluir la funcionalidad de extracción de informes útiles para departamentos como márketing como pudiera ser volumetrías de clientes dados de alta en un período determinado o productos más contratados.
\item Incluir un canal de comunicación (por ejemplo procesos batch que generan o cargan ficheros de datos) con los distintos sistemas con los que se comunica dentro del área comercial de la empresa, como pudieran ser los sistemas de provisión y facturación, de forma que exista un flujo de información que garantice una correcta alineación en los sistemas del estado de los datos implicados (contrataciones pendientes de provisión, finalización de provisión, suspensión por impago,\dots).
\item Incluir la posibilidad de presentar la aplicación en distintos idiomas mediante la internalización de java.
\end{itemize}
