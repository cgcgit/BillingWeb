%%%%%%%%%%%%%%%%%%%%%%%%%%%%%%%%%%%%%%%%%%%%%%%%%%%%%%%%%%%%%%%%%%%%%%%%%%%%%%%%
% Obxectivo: Lista de termos empregados no documento,                          %
%            xunto cos seus respectivos significados.                          %
%%%%%%%%%%%%%%%%%%%%%%%%%%%%%%%%%%%%%%%%%%%%%%%%%%%%%%%%%%%%%%%%%%%%%%%%%%%%%%%%

\newglossaryentry{bytecode}{
  name=bytecode,
  description={Código independente da máquina que xeran compiladores de determinadas linguaxes (Java, Erlang,\dots) e que é executado polo correspondente intérprete.}
}
\newglossaryentry{opensource}{
  name=código abierto,
  description={Modelo de desarrollo de software basado en la colaboración abierta que debe cumplir los siguientes requisitos: 
  \begin{itemize}
   \item Libre redistribución: el software debe poder ser regalado o vendido libremente.
   \item Código fuente: el código fuente debe estar incluido u obtenerse libremente.
   \item Trabajos derivados: la redistribución de modificaciones debe estar permitida.
   \item Integridad del código fuente del autor: las licencias pueden requerir que las modificaciones sean redistribuidas solo como parches.
   \item La licencia no debe discriminar a ninguna persona o grupo: nadie puede dejarse fuera.
   \item  Sin discriminación de áreas de iniciativa: los usuarios comerciales no pueden ser excluidos.
   \item  Distribución de la licencia: deben aplicarse los mismos derechos a todo el que reciba el programa.
   \item  La licencia no debe ser específica de un producto: el programa no puede licenciarse solo como parte de una distribución mayor.
   \item  La licencia no debe restringir otro software: la licencia no puede obligar a que algún otro software que sea distribuido con el software abierto deba también ser de código abierto.
   \item  La licencia debe ser tecnológicamente neutral: no debe requerirse la aceptación de la licencia por medio de un acceso por clic de ratón o de otra forma específica del medio de soporte del software.
  \end{itemize} 
     }
}