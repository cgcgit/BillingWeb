%%%%%%%%%%%%%%%%%%%%%%%%%%%%%%%%%%%%%%%%%%%%%%%%%%%%%%%%%%%%%%%%%%%%%%%%%%%%%%%%
% Obxectivo: Lista de siglas, abreviaturas, acrónimos, etc. empregados         %
%            no documento, xunto cos seus respectivos significados.            %
%%%%%%%%%%%%%%%%%%%%%%%%%%%%%%%%%%%%%%%%%%%%%%%%%%%%%%%%%%%%%%%%%%%%%%%%%%%%%%%%

\newacronym{erlang}{ERLANG/OTP}{Erlang Open Telecom Platform}
\newacronym{tfg}{TFG}{Trabajo fin de grado}
\newacronym{crm}{CRM}{Gestión de relaciones con el cliente o \textit{Customer Relationship Management}}
\newacronym{sge}{SGE}{Sistemas de gestión empresarial}
\newacronym{bpm}{BPM}{Gestión de procesos de negocio o \textit{Business Process Management}}
\newacronym{aris}{ARIS}{Arquitectura de sistema de información integrada o \textit{Architecture of Integrated Information Systems}}
\newacronym{oss}{OSS}{Sistemas de soporte operacional u \textit{Operational Support Systems}}
\newacronym{bss}{BSS}{Sistemas de soporte de negocio o \textit{Business Support Systems}}
\newacronym{itut}{ITU-T}{Sector de normalización de las telecomunicaciones de la UIT o \textit{ITU Telecommunication Standardization Sector}}
\newacronym{itu}{ITU}{Unión Internacional de Telecomunicaciones o \textit{International Telecommunication Union}}
\newacronym{ccitt}{CCITT}{Comité Consultivo Internacional de Telégrafos o \textit{Comité Consultatif International Téléphonique et Télégraphique}}
\newacronym{tmn}{TMN}{Red de Gestión de Telecomunicaciones o \textit{Telecommunications Management Network}}
\newacronym{tmf}{TM Forum}{Foro de Gestión de Telecomunicaciones o \textit{Telecommunication Management Forum}}
\newacronym{tic}{TIC}{Tecnologías de la Información y la Comunicación}
\newacronym{tam}{TAM}{Mapa de aplicaciones de telecomunicaciones o \textit{Telecom Applications Map}}
\newacronym{etom}{eToM}{Mapa de Operación de Telecomunicaciones mejorado o \textit{enhanced Telecom Operation Map}}
\newacronym{sid}{SID}{Modelo de información/datos compartidos o \textit{Shared Information/Data model}}
\newacronym{tna}{TNA}{Arquitectura tecnológica neutral o \textit{Technology Neutral Architecture}}
\newacronym{tip}{TIP}{Programa de integración del foro TM o \textit{TM Forum Integration Project}}
\newacronym{oda}{ODA}{TM Forum Open Digital Architecture}
\newacronym{itsm}{ITSM}{Gestión de servicios de tecnología de la información o \textit{Information Technology Service Management}}
\newacronym{itil}{ITIL}{Information Technology Infrastructure Library o \textit{Information Technology Infrastructure Library}}
\newacronym{ti}{TI}{Tecnología de la Información}
\newacronym{sfa}{SFA}{Automatización de la fuerza de ventas o \textit{Sales Force Automation}}
\newacronym{erp}{ERP}{Software de planificación de recursos empresariales o \textit{Enterprise resource planning software}}
\newacronym{jee}{Jakarta EE}{Jakarta Entreprise Edition}
\newacronym{jse}{Java SE}{Java Standard Edition}
\newacronym{jvm}{JVM}{Máquina virtual de java o \textit{Java Virtual Machine}}
\newacronym{api}{API}{Interfaz de programación de aplicaciones o \textit{Application Programming Interface}}
\newacronym{html}{HTML}{Lenguaje de marcado de hipertexto o \textit{Hypertext Markup Language}}
\newacronym{xml}{XML}{Lenguaje de marcado extensible o \textit{Extensible Markup Language}}
\newacronym{http}{HTTP}{Protocolo de transferencia de hipertexto o \textit{Hypertext Transfer Protocol} }
\newacronym{cdi}{CDI}{ontextos e inyección de dependencias o \textit{Contexts and Dependency Injection}}
\newacronym{el}{EL}{Lenguaje de expresiones o \textit{Expression Languaje}}
\newacronym{jsf}{JSF}{Jakarta Server Faces}
\newacronym{jsp}{JSP}{Jakarta Server Pages}
\newacronym{jcp}{JCP}{Java Community Process}
\newacronym{rest}{REST}{Transferencia de estado representativo o \textit{Representation State Transfer}}
\newacronym{json}{JSON}{JavaScript Object Notation o \textit{JavaScript Object Notation}}
\newacronym{soap}{SOAP}{Simple Object Access Protocol o \textit{Simple Object Access Protocol}}
\newacronym{ejb}{EJB}{Jakarta Enterprise Bean}
\newacronym{jpa}{JPA}{Jakarta Persistence API o \textit{Jakarta Persistence API}}
\newacronym{jta}{JTA}{API de transacción Jakarta o \textit{Jakarta Transaction API}}
\newacronym{jms}{JMS}{Jakarta Messaging}
\newacronym{eis}{EIS}{Sistemas de información empresarial o \textit{Enterprise Information Systems}}
\newacronym{eai}{EAI}{Integración de aplicaciones empresariales o \textit{Enterprise Application Integration}}
\newacronym{gui}{GUI}{Interfaz gráfica de usuario o \textit{Graphic User Interface}}
\newacronym{mvc}{MVC}{Modelo vista controlador o \textit{Model View Controller - MVC}}
\newacronym{jooq}{JOOQ}{Java Object Oriented Query}
\newacronym{dsl}{DSL}{Lenguaje de dominio específico o \textit{Domain Specific Languaje}}
\newacronym{ide}{IDE}{Entorno de desarrollo integrado o \textit{Integrated Development Environment}}
\newacronym{orm}{ORM}{Mapeo objeto-relacional o \textit{Object-Relational Mapping}}
\newacronym{sql}{SQL}{Lenguaje de consulta estructurada o \textit{Structured Query Language}}
\newacronym{pojo}{POJO}{Plain Old Java Object}
\newacronym{sgbd}{SGBD}{Sistema gestor de base de datos o \textit{Data Base Management System - DBMS}}
\newacronym{uml}{UML}{Lenguale de modelado unificado o \textit{Unified
 Modeling Language}}
 \newacronym{xhtml}{XHTML}{Lenguaje de marcado de hipertexto extensible o \textit{eXtensible HyperText Markup Language}}
\newacronym{comasw}{CoMaSw}{Contract Management Software}